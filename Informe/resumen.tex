\section{Resumen}
Este trabajo se centra en el estudio del algoritmo PageRank, propuesto en 1998 e implementado por el reconocido motor de búsqueda de Google, utilizado para clasificar sitios web por orden de relevancia. De acuerdo a Google: \\
\begin{displayquote}
PageRank funciona contando el n\'umero y la calidad de los enlaces a una p\'agina web para determinar una estimaci\'on aproximada de la importancia de la misma. La suposici\'on subyacente es que es probable que los sitios web m\'as importantes reciban m\'as enlaces de otros sitios web. \\
\end{displayquote}
	En el presente trabajo, se encontrar\'an los detalles de una posible implementaci\'on, incluyendo la elecci\'on de una estructura de datos apropiada para trabajar eficientemente con una gran cantidad de datos, bajo la asunci\'on de ralidad. \\
	También se llevar\'a a cabo un exhaustivo an\'alisis de la \textit{performance} del algoritmo, teniendo en cuenta tanto el tiempo de c\'omputo, como la calidad de los resultados que produce. \\
	Adem\'as se lo utilizar\'a para modelar distintos escenarios considerados de inter\'es cualitativo, y posteriormente se analizar\'a lo apropiado de los resultados.